\documentclass[]{article}
\usepackage{lmodern}
\usepackage{amssymb,amsmath}
\usepackage{ifxetex,ifluatex}
\usepackage{fixltx2e} % provides \textsubscript
\ifnum 0\ifxetex 1\fi\ifluatex 1\fi=0 % if pdftex
  \usepackage[T1]{fontenc}
  \usepackage[utf8]{inputenc}
\else % if luatex or xelatex
  \ifxetex
    \usepackage{mathspec}
  \else
    \usepackage{fontspec}
  \fi
  \defaultfontfeatures{Ligatures=TeX,Scale=MatchLowercase}
\fi
% use upquote if available, for straight quotes in verbatim environments
\IfFileExists{upquote.sty}{\usepackage{upquote}}{}
% use microtype if available
\IfFileExists{microtype.sty}{%
\usepackage{microtype}
\UseMicrotypeSet[protrusion]{basicmath} % disable protrusion for tt fonts
}{}
\usepackage[margin=1in]{geometry}
\usepackage{hyperref}
\hypersetup{unicode=true,
            pdfkeywords={true},
            pdfborder={0 0 0},
            breaklinks=true}
\urlstyle{same}  % don't use monospace font for urls
\usepackage{color}
\usepackage{fancyvrb}
\newcommand{\VerbBar}{|}
\newcommand{\VERB}{\Verb[commandchars=\\\{\}]}
\DefineVerbatimEnvironment{Highlighting}{Verbatim}{commandchars=\\\{\}}
% Add ',fontsize=\small' for more characters per line
\usepackage{framed}
\definecolor{shadecolor}{RGB}{248,248,248}
\newenvironment{Shaded}{\begin{snugshade}}{\end{snugshade}}
\newcommand{\AlertTok}[1]{\textcolor[rgb]{0.94,0.16,0.16}{#1}}
\newcommand{\AnnotationTok}[1]{\textcolor[rgb]{0.56,0.35,0.01}{\textbf{\textit{#1}}}}
\newcommand{\AttributeTok}[1]{\textcolor[rgb]{0.77,0.63,0.00}{#1}}
\newcommand{\BaseNTok}[1]{\textcolor[rgb]{0.00,0.00,0.81}{#1}}
\newcommand{\BuiltInTok}[1]{#1}
\newcommand{\CharTok}[1]{\textcolor[rgb]{0.31,0.60,0.02}{#1}}
\newcommand{\CommentTok}[1]{\textcolor[rgb]{0.56,0.35,0.01}{\textit{#1}}}
\newcommand{\CommentVarTok}[1]{\textcolor[rgb]{0.56,0.35,0.01}{\textbf{\textit{#1}}}}
\newcommand{\ConstantTok}[1]{\textcolor[rgb]{0.00,0.00,0.00}{#1}}
\newcommand{\ControlFlowTok}[1]{\textcolor[rgb]{0.13,0.29,0.53}{\textbf{#1}}}
\newcommand{\DataTypeTok}[1]{\textcolor[rgb]{0.13,0.29,0.53}{#1}}
\newcommand{\DecValTok}[1]{\textcolor[rgb]{0.00,0.00,0.81}{#1}}
\newcommand{\DocumentationTok}[1]{\textcolor[rgb]{0.56,0.35,0.01}{\textbf{\textit{#1}}}}
\newcommand{\ErrorTok}[1]{\textcolor[rgb]{0.64,0.00,0.00}{\textbf{#1}}}
\newcommand{\ExtensionTok}[1]{#1}
\newcommand{\FloatTok}[1]{\textcolor[rgb]{0.00,0.00,0.81}{#1}}
\newcommand{\FunctionTok}[1]{\textcolor[rgb]{0.00,0.00,0.00}{#1}}
\newcommand{\ImportTok}[1]{#1}
\newcommand{\InformationTok}[1]{\textcolor[rgb]{0.56,0.35,0.01}{\textbf{\textit{#1}}}}
\newcommand{\KeywordTok}[1]{\textcolor[rgb]{0.13,0.29,0.53}{\textbf{#1}}}
\newcommand{\NormalTok}[1]{#1}
\newcommand{\OperatorTok}[1]{\textcolor[rgb]{0.81,0.36,0.00}{\textbf{#1}}}
\newcommand{\OtherTok}[1]{\textcolor[rgb]{0.56,0.35,0.01}{#1}}
\newcommand{\PreprocessorTok}[1]{\textcolor[rgb]{0.56,0.35,0.01}{\textit{#1}}}
\newcommand{\RegionMarkerTok}[1]{#1}
\newcommand{\SpecialCharTok}[1]{\textcolor[rgb]{0.00,0.00,0.00}{#1}}
\newcommand{\SpecialStringTok}[1]{\textcolor[rgb]{0.31,0.60,0.02}{#1}}
\newcommand{\StringTok}[1]{\textcolor[rgb]{0.31,0.60,0.02}{#1}}
\newcommand{\VariableTok}[1]{\textcolor[rgb]{0.00,0.00,0.00}{#1}}
\newcommand{\VerbatimStringTok}[1]{\textcolor[rgb]{0.31,0.60,0.02}{#1}}
\newcommand{\WarningTok}[1]{\textcolor[rgb]{0.56,0.35,0.01}{\textbf{\textit{#1}}}}
\usepackage{graphicx,grffile}
\makeatletter
\def\maxwidth{\ifdim\Gin@nat@width>\linewidth\linewidth\else\Gin@nat@width\fi}
\def\maxheight{\ifdim\Gin@nat@height>\textheight\textheight\else\Gin@nat@height\fi}
\makeatother
% Scale images if necessary, so that they will not overflow the page
% margins by default, and it is still possible to overwrite the defaults
% using explicit options in \includegraphics[width, height, ...]{}
\setkeys{Gin}{width=\maxwidth,height=\maxheight,keepaspectratio}
\IfFileExists{parskip.sty}{%
\usepackage{parskip}
}{% else
\setlength{\parindent}{0pt}
\setlength{\parskip}{6pt plus 2pt minus 1pt}
}
\setlength{\emergencystretch}{3em}  % prevent overfull lines
\providecommand{\tightlist}{%
  \setlength{\itemsep}{0pt}\setlength{\parskip}{0pt}}
\setcounter{secnumdepth}{0}
% Redefines (sub)paragraphs to behave more like sections
\ifx\paragraph\undefined\else
\let\oldparagraph\paragraph
\renewcommand{\paragraph}[1]{\oldparagraph{#1}\mbox{}}
\fi
\ifx\subparagraph\undefined\else
\let\oldsubparagraph\subparagraph
\renewcommand{\subparagraph}[1]{\oldsubparagraph{#1}\mbox{}}
\fi

%%% Use protect on footnotes to avoid problems with footnotes in titles
\let\rmarkdownfootnote\footnote%
\def\footnote{\protect\rmarkdownfootnote}

%%% Change title format to be more compact
\usepackage{titling}

% Create subtitle command for use in maketitle
\providecommand{\subtitle}[1]{
  \posttitle{
    \begin{center}\large#1\end{center}
    }
}

\setlength{\droptitle}{-2em}

  \title{true}
    \pretitle{\vspace{\droptitle}\centering\huge}
  \posttitle{\par}
    \author{true \\ true}
    \preauthor{\centering\large\emph}
  \postauthor{\par}
    \date{}
    \predate{}\postdate{}
  

\begin{document}
\maketitle
\begin{abstract}
The abstract of the article.
\end{abstract}

\begin{Shaded}
\begin{Highlighting}[]
\KeywordTok{library}\NormalTok{(shiny)}
\end{Highlighting}
\end{Shaded}

\hypertarget{introduction}{%
\section{Introduction}\label{introduction}}

This tutorial provides examples of the functions provided within the
\pkg{nprcmanager} package.

\hypertarget{code-formatting}{%
\subsection{Code formatting}\label{code-formatting}}

Don't use markdown, instead use the more precise latex commands:

\begin{itemize}
\item
  \proglang{Java}
\item
  \pkg{plyr}
\item
  \code{print("abc")}
\end{itemize}

\hypertarget{r-code}{%
\section{R code}\label{r-code}}

Can be inserted in regular R markdown blocks.

\begin{Shaded}
\begin{Highlighting}[]
\NormalTok{x <-}\StringTok{ }\DecValTok{1}\OperatorTok{:}\DecValTok{10}
\NormalTok{x}
\end{Highlighting}
\end{Shaded}

\begin{verbatim}
##  [1]  1  2  3  4  5  6  7  8  9 10
\end{verbatim}

\hypertarget{importing-in-the-pedigree-data}{%
\section{Importing in the Pedigree
Data}\label{importing-in-the-pedigree-data}}

\hypertarget{pedigree-file-structure}{%
\subsection{Pedigree File Structure}\label{pedigree-file-structure}}

\begin{Shaded}
\begin{Highlighting}[]
\KeywordTok{includeHTML}\NormalTok{(}\StringTok{"../extdata/input_format.html"}\NormalTok{)}
\end{Highlighting}
\end{Shaded}

Input File Handling:

This software requires an input file for each breeding colony to be
analyzed. Multiple separate pedigrees within the same breeding colony
can be accommodated within the same input file.

If an animal is missing information on one parent, that cell should be
blank. For animals with no information on either parent, both cells
under ``Sire ID'' and ``Dam ID'' should be blank. For animals with
information on one parent a unique ``placeholder'' ID (designated as
UnkownID and UnkID in Pedigree Brower tab) will be assigned by the
software to represent the unknown parent. Please be aware that animals
with no parents will be treated as founders in these calculations, i.e.,
sources of new genetic variation in the colony.

Genotypic information for a single locus can be included in the genetic
analyses. See Pedigree File Format and Genotype File Format below for
file descriptions. Observed genotypic information can be supplied for
any members of the pedigree. Designation of two alleles is required as
currently there is no accommodation for partial genetic information for
an individual. Please contact R. Mark Sharp if you need to be able to
enter partial genetic information.

Pedigree File Format Details:

The following columns names are reserved and will be recognized:

Ego ID, Sire ID, Dam ID, and Sex are required columns.

Either Age or Birth must also be provided.

Use the Allowable Names listed in the table below. Allowable Name case
and order are not significant. The first Allowable Name listed for each
column is the name used internally and is the name used in examples and
output.

If the Age column is provided, the program will use the user-specified
age.

Otherwise, the program will calculate age based on the Birth column and
the current date, or the Death, Departure, or Exit columns, if provided.

Exit dates, if provided, will supercede Death and Departure dates. If
Exit is not provided, Death and Departure will be combined based on
which date is chronologically first. (This functions primarily to catch
clerical errors or historical changes in record-keeping practices).

Dates should be in the format YYYY-MM-DD (this date option can be found
under the ``English-U.K.'' locale option under the ``Format Cells
-\textgreater{} Number -\textgreater{} Date'' menu in Excel 2010).

Genotype data may be supplied within the pedigree file or in a separate
genotype file. Only two additional columns (first and second) are
required when the genotypes are provided within the pedigree file. The
columns are described below.

Allowable Name

\begin{verbatim}
<th>Description of Information</th>
    <th>Data Format</th>
</tr>
<tr bgcolor="#CCCCFF">
    <td>id, egoid, ego_id</td>
    <td>Ego ID: Unique animal identifier</td>
    <td>Alphanumeric characters (no symbols)</td>
</tr>
<tr bgcolor="#CCCCFF">
    <td>sire, sireid, sire_id</td>
    <td>Sire ID: Unique identifier of the ego's father</td>
    <td>Alphanumeric characters (no symbols)</td>
</tr>
<tr bgcolor="#CCCCFF">
    <td>dam, damid, dam_id</td>
    <td>Dam ID: Unique identifier of the ego's mother</td>
    <td>Alphanumeric characters (no symbols)</td>
</tr>
<tr bgcolor="#CCCCFF">
    <td>sex</td>
    <td>Sex: Ego's sex</td>
    <td>
        Sex can be indicated by any of the following:
        <br>
        Male: ("male", "m", "1"),
        Female: ("female", "f", "2"),
        Unknown: ("unknown", "u", "3"),
        Hermaphrodite: ("hermaphrodite", "h", "4")
    </td>
</tr>
<tr bgcolor="#CCFFFF">
    <td>age</td>
    <td>Age: Age at exit, or current age of the ego.</td>
    <td>Age in decimal years</td>
</tr>
<tr bgcolor="#CCFFFF">
    <td>birth, birthdate, birth_date</td>
    <td>Birth: Ego's date of birth</td>
    <td>YYYY-MM-DD Format</td>
</tr>
<tr>
    <td>death</td>
    <td>Death: Ego's date of death</td>
    <td>YYYY-MM-DD Format</td>
</tr>
<tr>
    <td>departure</td>
    <td>Departure: Ego's date of sale</td>
    <td>YYYY-MM-DD Format</td>
</tr>
<tr>
    <td>exit</td>
    <td>Exit: Ego's date of exit from the colony.
        <br>
        (Supercedes Death or Departure information)
    </td>
    <td>YYYY-MM-DD Format</td>
</tr>
<tr bgcolor="#FSFFCC">
    <td>allele_1</td>
    <td>Allele 1: Alphanumeric representation of first allele</td>
    <td>Currently limited to alphanumeric values, underscores, spaces and
    dashes. Other characters have not been tested.
    </td>
</tr>
<tr bgcolor="#F2FFCC">
    <td>allele_2</td>
    <td>Allele 2: Alphanumeric representation of second allele</td>
    <td>Currently limited to alphanumeric values, underscores, spaces and
    dashes. Other characters have not been tested.
    </td>
</tr>
\end{verbatim}

You will need to save your input file as a plain text file (either .txt
or .csv), in order to upload it correctly. To do this, you may use the
``Save As'' function in Excel to save a spreadsheet in a
``tab-delimited'' or ``comma-delimited'' file format.

Select the correct file type, and then click on ``Select Input File'' to
upload the file.

The pedigree information displayed in the Pedigree Browser will not
appear exactly the same as the input file. The pedigree shown in the
Pedigree Browser has been built using input file contents that may be
used to produce other columns of information displayed. Input file
processing will include:

A check for the 4 required fields: Ego ID, Sire ID, Dam ID, and Sex.

A check for either Age or Birth information.

A new row entry will be added for any Sire or Dam that do not already
have their own row as an Ego.

Animals will be checked to ensure that their sex is consistent
throughout the file.

Duplicate rows will be removed.

If Death or Departure date columns have been included, an ``Exit''
column will be built. Otherwise, Exit date assignment will be skipped if
an Exit date column was provided in the input file.

If Birth dates have been included, age will be calculated as (Birth Date
- Exit Date) / 365.25. If no Death, Departure, or Exit dates are
provided, it is assumed that the ego is still alive, and Age will be
calculated as (Birth Date - Today's Date) / 365.25. Age calculation will
be skipped if an Age information column is included in the input file.

Generation numbers will be added to each ego.

Families (Pedigrees) will be numbered and this information will be added
to each ego.

Genotype File Format Details:

The following columns names are reserved and will be recognized:

Columns id, first\_name, and second\_name are required columns.

Columns first and second will be automatically generated and cannot be
present.

.

Use the Allowable Names listed in the table below. Allowable Name case
and order are not significant. The Allowable Name listed for each column
is the name used internally and is the name used in examples and output.

Genotype data can be supplied within the pedigree file or in this
genotype file. Only two additional columns (first and second are
required when the genotypes are provided within the pedigree file. See
table above for a description. The genotype file columns are described
below.

Allowable Name

Description of Information

Data Format

id

ID: Unique animal identifier

Alphanumeric characters (no symbols)

allele\_1

Allele 1: Alphanumeric representation of first allele

Currently limited to alphanumeric values, underscores, spaces and
dashes.

allele\_2

Allele 2: Alphanumeric representation of second allele

Currently limited to alphanumeric values, underscores, spaces and
dashes. Other characters have not been tested.

You will need to save your genotype file as a plain text file (either
.txt or .csv), in order to upload it correctly. To do this, you may use
the ``Save As'' function in Excel to save a spreadsheet in a
``tab-delimited'' or ``comma-delimited'' file format.

Select the correct file type, and then click on ``Select Genotype File''
to upload the file.

It begins with importing a simple pedigree file, pedigree file with
genotypes, pedigree file and and associated set of genotypes, and a
initiating a database query that returns the full pedigree associated
with a list of animals provided by the user.

The recommended first step after creation of an input file is checking
the file for detectable errors.

\hypertarget{selecting-the-minimam-parental-age}{%
\subsection{Selecting the Minimam Parental
Age}\label{selecting-the-minimam-parental-age}}

\hypertarget{pedigree-file-only}{%
\subsection{Pedigree File Only}\label{pedigree-file-only}}

The field seperator used within a pedigree file must be the same
throughout and must be indicated when reading in the file. Fields can be
separated by tabs, commas, or simicolons.

\hypertarget{pedigree-and-genotypes-in-the-same-file}{%
\subsection{Pedigree and Genotypes in the Same
File}\label{pedigree-and-genotypes-in-the-same-file}}

\hypertarget{pedigree-and-genotypes-in-separate-files}{%
\subsection{Pedigree and Genotypes in Separate
Files}\label{pedigree-and-genotypes-in-separate-files}}

\hypertarget{pedigree-provided-by-labkey-query}{%
\subsection{Pedigree Provided by LabKey
Query}\label{pedigree-provided-by-labkey-query}}


\end{document}
