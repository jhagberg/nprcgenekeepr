% Options for packages loaded elsewhere
\PassOptionsToPackage{unicode}{hyperref}
\PassOptionsToPackage{hyphens}{url}
%
\documentclass[
]{article}
\usepackage{lmodern}
\usepackage{amssymb,amsmath}
\usepackage{ifxetex,ifluatex}
\ifnum 0\ifxetex 1\fi\ifluatex 1\fi=0 % if pdftex
  \usepackage[T1]{fontenc}
  \usepackage[utf8]{inputenc}
  \usepackage{textcomp} % provide euro and other symbols
\else % if luatex or xetex
  \usepackage{unicode-math}
  \defaultfontfeatures{Scale=MatchLowercase}
  \defaultfontfeatures[\rmfamily]{Ligatures=TeX,Scale=1}
\fi
% Use upquote if available, for straight quotes in verbatim environments
\IfFileExists{upquote.sty}{\usepackage{upquote}}{}
\IfFileExists{microtype.sty}{% use microtype if available
  \usepackage[]{microtype}
  \UseMicrotypeSet[protrusion]{basicmath} % disable protrusion for tt fonts
}{}
\makeatletter
\@ifundefined{KOMAClassName}{% if non-KOMA class
  \IfFileExists{parskip.sty}{%
    \usepackage{parskip}
  }{% else
    \setlength{\parindent}{0pt}
    \setlength{\parskip}{6pt plus 2pt minus 1pt}}
}{% if KOMA class
  \KOMAoptions{parskip=half}}
\makeatother
\usepackage{xcolor}
\IfFileExists{xurl.sty}{\usepackage{xurl}}{} % add URL line breaks if available
\IfFileExists{bookmark.sty}{\usepackage{bookmark}}{\usepackage{hyperref}}
\hypersetup{
  pdftitle={Meeting Notes and Things To Do},
  pdfauthor={R. Mark Sharp},
  hidelinks,
  pdfcreator={LaTeX via pandoc}}
\urlstyle{same} % disable monospaced font for URLs
\usepackage[margin=1in]{geometry}
\usepackage{longtable,booktabs}
% Correct order of tables after \paragraph or \subparagraph
\usepackage{etoolbox}
\makeatletter
\patchcmd\longtable{\par}{\if@noskipsec\mbox{}\fi\par}{}{}
\makeatother
% Allow footnotes in longtable head/foot
\IfFileExists{footnotehyper.sty}{\usepackage{footnotehyper}}{\usepackage{footnote}}
\makesavenoteenv{longtable}
\usepackage{graphicx,grffile}
\makeatletter
\def\maxwidth{\ifdim\Gin@nat@width>\linewidth\linewidth\else\Gin@nat@width\fi}
\def\maxheight{\ifdim\Gin@nat@height>\textheight\textheight\else\Gin@nat@height\fi}
\makeatother
% Scale images if necessary, so that they will not overflow the page
% margins by default, and it is still possible to overwrite the defaults
% using explicit options in \includegraphics[width, height, ...]{}
\setkeys{Gin}{width=\maxwidth,height=\maxheight,keepaspectratio}
% Set default figure placement to htbp
\makeatletter
\def\fps@figure{htbp}
\makeatother
\setlength{\emergencystretch}{3em} % prevent overfull lines
\providecommand{\tightlist}{%
  \setlength{\itemsep}{0pt}\setlength{\parskip}{0pt}}
\setcounter{secnumdepth}{-\maxdimen} % remove section numbering

\title{Meeting Notes and Things To Do}
\author{R. Mark Sharp}
\date{01/06/2020}

\begin{document}
\maketitle

Provided in reverse chronological order

\hypertarget{section}{%
\subsection{20200106}\label{section}}

\begin{enumerate}
\def\labelenumi{\arabic{enumi}.}
\tightlist
\item
  Mark will convert nprcmanager to nprcgenekeepr.
\item
  Amanda will provide notes for the Shiny application tutorial
\item
  Amanda will talk to Dave Lawrence about using his code to guide R
  development of the \emph{Diversity Report}.
\item
  Mark will complete submission to CRAN for publication to the
  consortium.
\item
  Mark has completed only the first of the four items from the 20191120
  meeting.
\item
  Expect that the package will be accepted on CRAN once examples have
  been provided for all exported functions. This is not a trivial matter
  as there 142 exported functions. I will be working to remove
\end{enumerate}

\hypertarget{section-1}{%
\subsection{20191218}\label{section-1}}

\begin{enumerate}
\def\labelenumi{\arabic{enumi}.}
\tightlist
\item
  Added ability to export each of the six figures on the \textbf{Summary
  Statistics} page.
\item
  Need to update shiny application tutorial to reflect the ability to
  export each of the six figures on the \textbf{Summary Statistics}
  page.
\end{enumerate}

\hypertarget{section-2}{%
\subsection{20191130}\label{section-2}}

\begin{enumerate}
\def\labelenumi{\arabic{enumi}.}
\tightlist
\item
  See if
  \url{https://www.semanticscholar.org/paper/Management-of-genetic-diversity-using-gene-dropping-Khaldari-Javaremi/43dd1975193830b9243c8eef672db134e42e871d/figure/3}
  is figure that would be beneficial.
\end{enumerate}

\hypertarget{section-3}{%
\subsection{20191120}\label{section-3}}

\begin{enumerate}
\def\labelenumi{\arabic{enumi}.}
\tightlist
\item
  Export all graphics
\item
  Create, error check and export dawn of time pedigree
\item
  Add heuristic to removed recent progeny as founders.

  \begin{enumerate}
  \def\labelenumii{\alph{enumii}.}
  \tightlist
  \item
    Add a column - something like ``from center'' so that ``born at
    site'' can be ascertained.
  \end{enumerate}
\item
  Provide a way to clear focal animals added in \textbf{Pedigree
  Browser} tab so that full pedigree is again availabe. This will also
  allow a new pedigree to be added via the \textbf{Input} tab.
\end{enumerate}

\hypertarget{section-4}{%
\subsection{20191115}\label{section-4}}

\begin{enumerate}
\def\labelenumi{\arabic{enumi}.}
\tightlist
\item
  Clear out empty text field (\emph{Filter View}) on \_Genetic Value
  Analysis\_\_ tab. Done 20191115
\end{enumerate}

\hypertarget{section-5}{%
\subsection{20191111}\label{section-5}}

\begin{enumerate}
\def\labelenumi{\arabic{enumi}.}
\tightlist
\item
  Send example pedigrees with errors to Amanda.Done 20191111
\item
  To meet Friday at 4 PM Pacific to go through entire presentation.
\item
  Code changes since August 16, 2016.

  \begin{itemize}
  \tightlist
  \item
    Initial package construction was in March of 2017.
  \item
    Start of code additions in August 2017.
  \item
    There are now 736 unit tests covering over 92\% of the code.
  \item
    Two tutorials have been developed. One to guide R users in the
    interactive use of the package functions (public API) and one to
    guide users of the Shiny application.
  \item
    Available via GitHub with MIT license.
  \end{itemize}
\end{enumerate}

\begin{longtable}[]{@{}lrrrrrr@{}}
\caption{Simple metrics for the Original Version (20160816) and Version
0.5.42.9001 (20191221).}\tabularnewline
\toprule
Source & Files & Lines & Code & Blank Lines & Document Lines & Comment
Lines\tabularnewline
\midrule
\endfirsthead
\toprule
Source & Files & Lines & Code & Blank Lines & Document Lines & Comment
Lines\tabularnewline
\midrule
\endhead
Original & 8 & 3621 & 1927 & 531 & 0 & 1163\tabularnewline
nprcmanager 0.37 & 285 & 11620 & 7375 & 439 & 2947 & 859\tabularnewline
\bottomrule
\end{longtable}

\hypertarget{section-6}{%
\subsection{20191014}\label{section-6}}

\begin{enumerate}
\def\labelenumi{\arabic{enumi}.}
\item
  \textbf{Progress Since Last Meeting}

  \begin{enumerate}
  \def\labelenumii{\alph{enumii}.}
  \tightlist
  \item
    Code changes

    \begin{itemize}
    \tightlist
    \item
      Added filter to pedigree IDs available for breeding group
      formation so that only animals at the institution (exit == NA) and
      animals with recorded birth dates (birth != NA) are potential
      breeding group members.
    \end{itemize}
  \item
    Documentation Updates

    \begin{itemize}
    \tightlist
    \item
      Shiny application tutorial first draft is nearly complete.
      Breeding group formation remains.
    \end{itemize}
  \item
    CRAN submission preparation

    \begin{itemize}
    \tightlist
    \item
      Have begun using RHUB tools to prepare for the CRAN submission.

      \begin{itemize}
      \tightlist
      \item
        library(rhub)
      \item
        library(usethis)
      \item
        cran\_prep \textless- check\_for\_cran()
      \item
        cran\_prep\$cran\_summary()
      \item
        usethis::use\_cran\_comments()
      \end{itemize}
    \end{itemize}
  \end{enumerate}
\item
  \textbf{Meeting Notes}

  \begin{enumerate}
  \def\labelenumii{\alph{enumii}.}
  \tightlist
  \item
    Mark will fly up on 19th leave 21st
  \item
    Need to ensure access to WiFi and video.
  \item
    Finish Shiny tutorial
  \item
    Recheck all files to ensure animals are deidentified via
    obfuscations of IDs, birth dates, and exit dates.
  \item
    Plan to have a CRAN submission prior to November 20, 2019, meeting.
  \item
    Want to develop a better name. This should be done prior to CRAN
    submission.
  \item
    Matt is to test LabKey connection.
  \item
    Amanda and Mark will see if Wayne can move forward on Mark's access
    to PRIMe
  \item
    Mark will work on \textbf{Genetic Production}, \textbf{Genetic Value
    Analysis}, and \textbf{Founders} as described in the meeting notes
    for 20190916.
  \end{enumerate}
\end{enumerate}

\hypertarget{section-7}{%
\subsection{20190916}\label{section-7}}

\begin{enumerate}
\def\labelenumi{\arabic{enumi}.}
\item
  \textbf{Production Calculation}

  Amanda corrected and clarified how to calculate \emph{Production}
  using the

  \begin{enumerate}
  \def\labelenumii{\alph{enumii}.}
  \tightlist
  \item
    The Production Status is calculated on September 09, 2019.
  \item
    Births = count of all animals in group born since January 1, 2017
    through December 31, 2018, that lived at least 30 days. Animals born
    after December 31, 2018, are not counted.
  \item
    Dams = count of all females in group that have a birth date on or
    prior to September 09, 2016.
  \item
    Production = Births / Dams
  \item
    Production Status (color)

    \begin{enumerate}
    \def\labelenumiii{\arabic{enumiii}.}
    \tightlist
    \item
      Shelter and pens

      \begin{enumerate}
      \def\labelenumiv{\arabic{enumiv}.}
      \tightlist
      \item
        Red -- \textless{} 0.51
      \item
        Yellow -- \textgreater= 0.51 \& \textless{} 0.54
      \item
        Green -- \textgreater= 0.54
      \end{enumerate}
    \item
      Corrals

      \begin{enumerate}
      \def\labelenumiv{\arabic{enumiv}.}
      \tightlist
      \item
        Red -- \textless{} 0.61
      \item
        Yellow -- \textgreater= 0.61 \& \textless{} 0.65
      \item
        Green -- \textgreater= 0.65
      \end{enumerate}
    \end{enumerate}
  \end{enumerate}
\item
  \textbf{Genetic Production}

  Percent of females \textgreater= 3.5 at the start of the 2 calendar
  year period that have not produced offspring in the past 2 calendar
  years. Filter out animals over ?? (start with 18) years old.
\item
  \textbf{Genetic Value Analysis} Have Mark look at algorithm and
  develop simple options for handling progeny that are missing parental
  information.

  One option is to not calculate genetic value of animals \textless{}
  minParentAge.
\item
  \textbf{Founders} We need to get a better understanding of the issues
  surrounding the identification of founders and how founders affect the
  genetic value analysis. As a first step Mark will develop code to list
  the founders and the count of male and female founders. See how they
  affect the genetic value analysis.

  In the very short table below \textbf{x123} has a founder as a dam
  while \textbf{x122} does not have a founder for a sire because of the
  respective values of \emph{dam\_from\_center} and
  \emph{sire\_from\_center}.
\end{enumerate}

\begin{longtable}[]{@{}llllll@{}}
\toprule
id & sire & sire\_from\_center & dam & dam\_from\_center &
birth\tabularnewline
\midrule
\endhead
x123 & s123 & Y & & N & 1988/04/21\tabularnewline
x122 & & Y & d123 & Y & 1989/08/18\tabularnewline
\bottomrule
\end{longtable}

\begin{enumerate}
\def\labelenumi{\arabic{enumi}.}
\setcounter{enumi}{4}
\item
  \textbf{Matt Schultz and R Setup for nprcmanager}

  Mark is to contact Matt to arrange a time to work with him to make
  sure he get nprcmanger set up as a repository on his computer.

  Mark will also send the interactive tutorial to him and Amanda. Done
  20190916.
\end{enumerate}

\hypertarget{section-8}{%
\subsection{20190826}\label{section-8}}

\begin{enumerate}
\def\labelenumi{\arabic{enumi}.}
\item
  Genetic Diversity Graphic

  \begin{enumerate}
  \def\labelenumii{\alph{enumii}.}
  \tightlist
  \item
    Labels Done 20190908

    \begin{itemize}
    \tightlist
    \item
      Breeding Group
    \item
      High-Low - Value
    \item
      Indian Origin - Origin
    \item
      Fecundity - Production
    \item
      Kinship With Male - Inbreeding
    \item
      Genotype Phenotype - Flags
    \end{itemize}
  \item
    Labels of groups on left
  \item
    Labels of genetic diversity labels top and 45 degree angle
  \item
    Genetic Diversity - Genetic Diversity Report
  \end{enumerate}
\item
  Send list of errors that can be detected. Done 20190826
\item
  Send rhesus MHC data pedigrees Done 20190826
\item
\end{enumerate}

\hypertarget{section-9}{%
\subsection{20190810}\label{section-9}}

This was an \emph{ad hoc} meeting to get clarification on heat map
development for genetic diversity reporting. This meeting was a response
to questions in a July 18, 2019, email from Mark in response to an email
from Amanda with the subject of ``genetic diversity reporting
questions''.

As a result of this meeting, the following columns are defined as
follows

\begin{enumerate}
\def\labelenumi{\arabic{enumi}.}
\tightlist
\item
  GENETIC DIVERSITY REPORTING

  \begin{enumerate}
  \def\labelenumii{\arabic{enumii}.}
  \tightlist
  \item
    What are the proportions of high and low genetic value breeding-age
    adults in the group?

    \begin{enumerate}
    \def\labelenumiii{\arabic{enumiii}.}
    \tightlist
    \item
      RED \textgreater{} 0.5 LOW VALUE
    \item
      0.5 \textgreater= YELLOW \textgreater= 0.30 LOW VALUE
    \item
      GREEN =\textless{} 0.3 LOW VALUE
    \end{enumerate}
  \item
    Are all members of the breeding group Indian-origin rhesus macaques?

    \begin{enumerate}
    \def\labelenumiii{\arabic{enumiii}.}
    \tightlist
    \item
      RED -- \textgreater= 1 HYBRID OR CHINESE ANIMALS
    \item
      YELLOW -- \textgreater= 1 BORDERLINE HYBRID ANCESTRY \& 0 HYBRID
      OR CHINESE ANIMALS
    \item
      GREEN -- 0 HYBRID OR CHINESE ANIMALS \& 0 BORDERLINE HYBRID
      ANCESTRY
    \end{enumerate}

    \begin{quote}
    = 10\% \textless= 15\% chinese: Proportion of chinese ancestry is
    borderline Yellow; Red \textgreater{} 15\%; This information is
    found under animal history under genetics main tab -- genetic
    ancestry.
    \end{quote}
  \item
    Among all females in group age \textgreater= 3 years what is
    percentage with a kinship coefficient \textless= 0.0156 with at
    least 1 male \textgreater= 5 years old in the group.

    \begin{enumerate}
    \def\labelenumiii{\arabic{enumiii}.}
    \tightlist
    \item
      Denomonator = count of females \textgreater= 3 years of age
    \item
      Numerator = count of females with kinship coefficient \textless=
      0.0156 with at least 1 male \textgreater= 5 years old in the
      group.
    \item
      Thresholds

      \begin{enumerate}
      \def\labelenumiv{\arabic{enumiv}.}
      \tightlist
      \item
        Red -- \textless{} 0.6
      \item
        Yellow -- \textgreater= 0.6 \& \textless= 0.9
      \item
        Green -- \textgreater{} 0.9
      \end{enumerate}
    \end{enumerate}
  \item
    Fecundity

    \begin{enumerate}
    \def\labelenumiii{\arabic{enumiii}.}
    \tightlist
    \item
      Denominator = count of females \textgreater= 3 years of age
    \item
      Numerator = count of births that live \textgreater{} 30 days
    \item
      Thresholds

      \begin{enumerate}
      \def\labelenumiv{\arabic{enumiv}.}
      \tightlist
      \item
        Shelter and pens

        \begin{enumerate}
        \def\labelenumv{\arabic{enumv}.}
        \tightlist
        \item
          Red -- \textless{} 0.6
        \item
          Yellow -- \textgreater= 0.6 \& \textless= 0.63
        \item
          Green -- \textgreater{} 0.63
        \end{enumerate}
      \item
        Corrals

        \begin{enumerate}
        \def\labelenumv{\arabic{enumv}.}
        \tightlist
        \item
          Red -- \textless{} 0.5
        \item
          Yellow -- \textgreater= 0.5 \& \textless= 0.53
        \item
          Green -- \textgreater{} 0.53
        \end{enumerate}
      \end{enumerate}
    \end{enumerate}
  \item
    Flagged for genotype of phenotype

    \begin{enumerate}
    \def\labelenumiii{\arabic{enumiii}.}
    \tightlist
    \item
      Thresholds

      \begin{enumerate}
      \def\labelenumiv{\arabic{enumiv}.}
      \tightlist
      \item
        Red -- \textgreater= 3 group members flagged
      \item
        Yellow -- \textless{} 3 and \textgreater= 1 group members
        flagged
      \item
        Green -- 0 group members flagged
      \end{enumerate}
    \end{enumerate}
  \end{enumerate}
\end{enumerate}

\hypertarget{section-10}{%
\subsection{20190715}\label{section-10}}

\begin{enumerate}
\def\labelenumi{\arabic{enumi}.}
\tightlist
\item
  To prepare for the meeting

  \begin{enumerate}
  \def\labelenumii{\arabic{enumii}.}
  \tightlist
  \item
    Work on tutorial to go through the Goldilocks path.
  \item
    Work on tutorial to look at all possible data input errors
  \item
    Submit new version to GitHub and Travis-ci
  \end{enumerate}
\item
  We discussed heat map type issues with regard to realtime examination
  of kinship relationships within breedeing groups

  \begin{enumerate}
  \def\labelenumii{\arabic{enumii}.}
  \tightlist
  \item
    Amanda is to send Mark a pedigree with breeding group information
    and an updated version of the questions in an Excel workbook
  \item
    Mark is to use those data and the questions to develop some
    visualizations for Amanda's review.
  \end{enumerate}
\item
  The interactive tutorial was discussed a bit.

  \begin{enumerate}
  \def\labelenumii{\arabic{enumii}.}
  \tightlist
  \item
    Priscilla Williams, a current graduate student, who Mark has worked
    with for several years is working through the tutorial.
  \item
    This discussion brought up the topic of making use of the Shiny
    application as simple as possible.
  \end{enumerate}
\item
  Mark is going to put together the Shiny application tutorial next
\end{enumerate}

\hypertarget{section-11}{%
\subsection{20190603}\label{section-11}}

\begin{enumerate}
\def\labelenumi{\arabic{enumi}.}
\item
  Joint Working Group Meeting week before Thanksgiving. We will meet
  with the Breeding Colony Managers Group. November 19-21, 2019.
\item
  Work on loop code: have list of animals involved in loops with counts,
  number of loops, number of animals in loops. Not high priority.
\item
  DCM wants to report on problems in current breeding groups prior
  occurance.

  \begin{itemize}
  \tightlist
  \item
    Amanda has developed an initial set of questions.

    \begin{itemize}
    \tightlist
    \item
      Are all Indian origin?
    \item
      Has alpha male been in group \textgreater{} 3 years.
    \item
      Are kinship coefficients of \textgreater=0.0156 between male and
      females \textgreater= 2.5 years old (settable) within a breeding
      group.
    \item
      Are kinship coefficients of \textgreater=0.0156 between males
      \textgreater= 2.5 years old (settable) within a breeding group.
    \item
      Are offspring equally distributed among female breeders.
    \item
      Are any members flagged to be genotyped. This likely does not
      reside in demographics.
    \end{itemize}
  \end{itemize}
\item
  Does Amanda know what the \emph{2014-10-16\_ResPed\_v1.1.txt},
  \emph{BreedingGroups1\_4MendozaTest.csv},
  \emph{Jmac\_studbook\_20180711.csv},
  \emph{Jmac\_studbook\_20180711.txt}, and
  \emph{MendozaC1C2newharemstest.csv} files represent? Can they be use
  for instruction? We are not going to use them. Done 20190622 Are some
  to be used with the Example\_Pedigree file? We are not going to use
  them. Done 20190622
\item
  Decide which pedigree files to leave in package. Done 20190622

  \begin{itemize}
  \tightlist
  \item
    All pedigree files except an obfuscated baboon (qcPed) and an
    obfuscated rhesus (examplePedigree) were moved out of the package.
  \item
    The qcPed and examplePedigree are in the package as data, which can
    be obtained by the user with

    \begin{itemize}
    \tightlist
    \item
      \textbf{qcPed \textless- nprcmanager::qcPed}
    \item
      \textbf{examplePedigree \textless- nprcmanager::examplePedigree}
    \end{itemize}
  \item
    Three versions of the qcPed pedigree were included as actual files
    for the getPedigree() unit tests.
  \end{itemize}
\item
  Figure out how to best provide example pedigree files. See point
  immediately above. Done 20190622
\item
  Items brought forward from previous meetings to prevent forgetting
  them.

  \begin{enumerate}
  \def\labelenumii{\arabic{enumii}.}
  \tightlist
  \item
    From 20190311

    \begin{itemize}
    \tightlist
    \item
      Amanda sent Mark the 2015-02-14\_Genetic\_metrics\_white
      paper\_Final.docx paper for him to review and collect ideas for a
      renewed ORIP Reporting tab. Amanda suggested that we need to
      propose collection of these metrics from all primate centers.
      Genetic and Genomics Working Group website:
      \url{https://nprcresearch.org/primate/genetics-genomics/genetics-genomics-working-group.php}
    \end{itemize}
  \item
    From 20190408

    \begin{itemize}
    \tightlist
    \item
      Consider adding the ability to ignore founders if the user selects
      that option.
    \item
      Go through all primary data structure names and definitions to
      update and correct.
    \item
      Edit all function descriptions to update and correct
    \end{itemize}
  \end{enumerate}
\end{enumerate}

\hypertarget{section-12}{%
\subsection{20190429}\label{section-12}}

\begin{enumerate}
\def\labelenumi{\arabic{enumi}.}
\tightlist
\item
  Updated Amanda's R, R packages, and installed version 0.5.10
  (20190428) of nprcmanager.
\item
  Demonstrated that new version of nprcmanager displayed suspicious
  parent table in ErrorTab. Done 20190428
\item
  Decided on new wording of ``One or both parents are below the minimum
  parental age. Check both parent and offspring birth dates.'' to place
  before the Suspicious Parents table. Done 20190428
\item
  Removed row label column in Suspicious Parents table. Done 20190428
\item
  Plan to add ability to read in pedigrees in Excel format to Input tab.
  Done 20190519
\item
  Still need to complete items listed in 20190408 meeting notes. (See
  below.)
\item
  Will provide a ``Goldilocks'' tutorial of Shiny application first.
\end{enumerate}

\hypertarget{section-13}{%
\subsection{20190408}\label{section-13}}

\begin{enumerate}
\def\labelenumi{\arabic{enumi}.}
\tightlist
\item
  Test for error when pedigree is not available when forming groups
\item
  Make sure harems are formed correctly when males are in seed animals
\item
  List of things found prior to the meeting to do before next meeting:

  \begin{itemize}
  \tightlist
  \item
    Unit test for fillGroupMembersWithSexRatio() Done 20190328
  \item
    Go through all primary data structure names and definitions to
    update and correct.
  \item
    Edit all function descriptions to update and correct
  \item
    Correct nprcmanager.R file. This includes the following and more.
    Check, correct, and create where needed function lists including
    lists of all functions, pedigree file testing functions, genetic
    value functions, plotting functions, breeding group formation
    functions, and gene dropping functions (if possible). Done 20190515
  \item
    Consider making a summary function for the gvReport structure. Done
    20190602
  \item
    Consider adding the ability to ignore founders if the user selects
    that option.
  \end{itemize}
\end{enumerate}

Ballou \& Lacy describe genome uniqueness as ``the proportion of
simulations in which an individual receives the only copy of a founder
allele.'' We have interpretted this as meaning that genome uniqueness
should only be calculated for living, non-founder animals. Alleles
possessed by living founders are not considered when calculating genome
uniqueness.

We have a differing view on this, since a living founder can still
contribute to the population. The function below calculates genome
uniqueness for all living animals and considers all alleles. It does not
ignore living founders and their alleles.

Our results for genome uniqueness will, therefore differ slightly from
those returned by Pedscope. Pedscope calculates genome uniqueness only
for non-founders and ignores the contribution of any founders in the
population. This will cause Pedscope's genome uniqueness estimates to
possibly be slightly higher for non-founders than what this function
will calculate.

\hypertarget{section-14}{%
\subsection{20190311}\label{section-14}}

\begin{enumerate}
\def\labelenumi{\arabic{enumi}.}
\tightlist
\item
  Remove explanatory text above sex ratio radio buttons. Done 20190311
\item
  Ignore kinship between females at or above the minimum parent age
  (Yes/No). Done 20190311.
\item
  Invoice for Jan 12 - Mar 11, 2019 sent 20190311.
\item
  Renamed (Pyramid Plot to Age Pyramid Plot) and moved to the right of
  Pedigree Browser. Done 20190311
\item
  Amanda sent Mark the 2015-02-14\_Genetic\_metrics\_white
  paper\_Final.docx paper for him to review and collect ideas for a
  renewed ORIP Reporting tab. Amanda suggested that we need to propose
  collection of these metrics from all primate centers. Genetic and
  Genomics Working Group website:
  \url{https://nprcresearch.org/primate/genetics-genomics/genetics-genomics-working-group.php}
\item
  Still have two features to add and one graphic element to correct.

  \begin{itemize}
  \tightlist
  \item
    Allow user to enter the number of groups of seed animals to be
    entered. See \emph{Seed Groups} from last month's notes. Done
    20190406
  \item
    Add \textbf{minParentAge} to check box for \emph{Use minimum parent
    age for age of animals to be grouped with the mother}. Done 20190406
  \end{itemize}
\item
  To meet again on 20190325 if Mark makes sufficient progress otherwise
  we will meet on 20190408 at 4PM PAC.
\end{enumerate}

\hypertarget{section-15}{%
\subsection{20190225}\label{section-15}}

\begin{enumerate}
\def\labelenumi{\arabic{enumi}.}
\tightlist
\item
  Mark to add additional unit tests for new seed group formation code.
\item
  Items to add to instructions:

  \begin{enumerate}
  \def\labelenumii{\alph{enumii}.}
  \tightlist
  \item
    Discuss what to do about overlapping group formation. Instruct the
    user to run the Make Groups command again to get a new set of
    groups. The new set will overlap with the prior results, but will be
    mutually exclussive within a run.
  \end{enumerate}
\item
  Mark to send an invoice at end of February. Amanda will send the
  amount of funding available. Done 20190311
\item
  Seed Groups

  \begin{itemize}
  \tightlist
  \item
    Have user enter number of seed groups to form and then present that
    number of windows.
  \item
    Change label to ``Optional: Seed Groups with Specific Animals''.
    Done 20190311
  \item
    Have it in its own gray box
  \end{itemize}
\item
  Pull downs to right in one column: Done 20190309

  \begin{enumerate}
  \def\labelenumii{\alph{enumii}.}
  \tightlist
  \item
    Number of groups desired
  \item
    Animals will be grouped with mother that are below age (years):
    {[}{]} Use minimum parent age. \textless-- default -- Have not
    figured out how to have \textbf{minParentAge} from Input tab to show
    up in the BreedingGroupFormation tab. Have dropdown disappear if
    checkbox above is selected.
  \item
    Animals with kinship above \ldots.
  \item
    Dropdown with

    \begin{itemize}
    \tightlist
    \item
      Ignore females at or above the minimum parent age
    \item
      Do not ignore females at or above the minimum parent age
    \end{itemize}
  \item
    Modify kinship dropdown to add common relationship nomenclature in
    parentheses.
  \item
    Number of simulations
  \end{enumerate}
\item
  Left column input: Done 20190309

  \begin{enumerate}
  \def\labelenumii{\alph{enumii}.}
  \tightlist
  \item
    Make Groups in color and larger
  \item
    Enter the group to view
  \item
    Export Current Group
  \item
    Export Current Group Kinship Matrix
  \end{enumerate}
\item
  Example emails

  \begin{enumerate}
  \def\labelenumii{\alph{enumii}.}
  \tightlist
  \item
    Example: Whether or not one of 10 females would work with a
    particular male.
  \item
    Amanda is to collect other similar emails to use as ideas for
    development of vignettes.
  \end{enumerate}
\item
  Next meeting March 11, 2019, at 4PM PAC.
\end{enumerate}

\hypertarget{section-16}{%
\subsection{20190114}\label{section-16}}

\begin{enumerate}
\def\labelenumi{\arabic{enumi}.}
\item
  Amanda provided a new logo to use, which represents all of the
  National Primate Research Centers. This is necessary because the ONPRC
  parent organization does not allow them to have a logo. Mark will
  incorporate that logo into the application and add some comments
  regarding development supported by ONPRC and SNPRC. Added 20190119.
\item
  Discussed harem formation and formation of groups with a set sex
  ratio. Done 20190103.
\item
  Discussed what to do about overlapping group formation. Decided to
  instruct the user to run the Make Groups command again to get a new
  set of groups. The new set will overlap with the prior results, but
  will be mutually exclussive within a run. Added to things to do for
  20190225 meeting 20190224
\item
  Discussed layout of breeding group formation tab by using a mockup
  tool. Decided to have six workflows selectable in a radio button
  configuration. The user interface will redraw based on the radio
  button selected to reflect the user interface elements needed within
  the selected workflow. Done 20190224
\item
  Need to test group formation with a candidate animal not in the
  pedigree or not in the genetic analysis. Found that this causes a
  application crash. Done 20190224
\item
  Need to test group formation with a candidate animal not in the
  pedigree or not in the genetic analysis. Found that groups are not
  formed, but an error is not displayed. The user must remove the false
  Id to get group formation to work. Done 20190224.
\item
  Make an error reporting function that informs the user what has
  occurred when an animal not in the pedigree or not in the genetic
  analysis is entered as a seed animal or a candidate animal.
\item
  Change ``Animals will be ignored below this age'' to ``Animals will be
  grouped with the mother below age''. Done 20190224.
\item
  Change to Animals with kinship above this value will be excluded. Done
  20190224.
\item
  Change to Include kinship in display of groups. Done 20190224.
\item
  Disable breeding group formation tab until genetic value analysis has
  occurred. Have disabled tab display text explaining that genetic value
  analysis must be performed first.
\item
  Have 6 boxes for seed animals. User can provide seed animals for up to
  six groups. Done 20190224
\item
  The next meeting will be 20190211 at 4 PM Pacific time.
\end{enumerate}

\hypertarget{section-17}{%
\subsection{20181210}\label{section-17}}

\begin{enumerate}
\def\labelenumi{\arabic{enumi}.}
\tightlist
\item
  Mark reported that he was having trouble implementing the user
  interface elements needed for items 4 and 5 of the 20181105 meeting
  notes. Specifically the need to have dynamically generated text input
  boxes for the number of groups to be formed is the issue. He has not
  figured out either the desired interface behavior or the technical
  methods needed for this feature. This is going to delay getting the
  tutorials written, which he had hoped to complete in first draft form
  during December.
\end{enumerate}

After some discussion, all agreed that the features described in items 4
and 5 are sufficiently important to delay the preparation of the
tutorials and subsequent technical paper a few weeks. Thus, Mark will
work on getting these implemented in December with plans on working on
the tutorials in January.

Ability to form harems. Done 20181230.

Ability to form breeding groups with specified sex ratios. Done
20180103.

\begin{enumerate}
\def\labelenumi{\arabic{enumi}.}
\tightlist
\item
  Amanda and Mark had fairly extensive discussions regarding how group
  selection is to be done. The current software forms groups so that
  there is no overlap of animals among any of the groups. The plan is
  for Mark to implement a group selection procedure that will make two
  types of groups.
\end{enumerate}

One group type is what is currently done by the software (no overlap of
animals among any of the groups formed.) These groups will be made up of
all of the animals that can be used in groups based on kinship and sex
ratio criteria such that none of the groups have any individuals in
common.

The second type of group is a collection of sub-groups where each
sub-group is a group of the first type. Thus, there is no overlap of
animals within any one sub-group of groups and there is potential
overlay among the various subgroups. This is complecated and hard to
follow so it is illustrated with the list below where each letter
represent a specific animal. Overlapping Groups 1-4 have within them
sets of animal that have no animals in common with other unique sets
within the group.

For example, animal \textbf{H} appears in set 1 of each overlapping
group, but it does not appear more than once in any on group. Also,
animal \textbf{E} appears in group 1, 2, and 4 and not in group 3.

\begin{itemize}
\tightlist
\item
  Overlapping Group 1 - Unique Set 1: I,B,J,L,H - Unique Set 2:
  D,G,U,S,Q - Unique Set 3: E,P,F,A,C
\item
  Overlapping Group 2 - Unique Set 1: M,V,H,Z,K - Unique Set 2:
  O,T,W,D,X - Unique Set 3: C,J,E,F,B
\item
  Overlapping Group 3 - Unique Set 1: H,I,F,X,K - Unique Set 2:
  Q,Y,A,D,S - Unique Set 3: T,B,C,G,M
\item
  Overlapping Group 4 - Unique Set 1: H,V,I,T,Z - Unique Set 2:
  E,W,M,N,O - Unique Set 3: P,Q,J,G,U
\end{itemize}

\begin{enumerate}
\def\labelenumi{\arabic{enumi}.}
\tightlist
\item
  Amanda and Mark are to develop a better descriptor for the button that
  causes the pedigree information to be read in tested and sent to the
  pedigree browser function. Mark has changed it to \emph{Read and Check
  Pedigree} for now. Done 20181212.
\item
  Mark is to reorganize the \textbf{Input} tab to place the description
  of the minimum parent age in hovertext and make the check pedigree
  button more evident. Done 20181212.
\item
  The next meeting will be 20190114 at 4 PM Pacific time.
\end{enumerate}

\hypertarget{section-18}{%
\subsection{20181105}\label{section-18}}

\begin{enumerate}
\def\labelenumi{\arabic{enumi}.}
\tightlist
\item
  Do not report as an error the wrong sex for animals added into the
  pedigree and appear as both a sire and dam without an ego record.
  These need to be reported as an error because they are both a sire and
  a dam. Done 20181208
\item
  Make a combined logo for Oregon and SNPRC. Have ONPRC on top using
  blue and green. Check on the University of Oregon website for a color
  palette. Have ONPRC in the lighter color similar to the SNPRC color
  scheme. Have the macaque and oval in the same blue color. Clean up the
  resolution of the Oregon logo as it is currently fuzzy. Some research
  found the following link
  \href{https://www.ohsu.edu/xd/digital-identity/global-elements-design/color-palette.cfm}{OHSU
  Color Palette}.
\end{enumerate}

\begin{itemize}
\tightlist
\item
  Alert \#C34D36
\item
  Body text \#555555
\item
  Button text \#805B16
\item
  Button pre-fade \#FFD769
\item
  Button post-fade \#FFC529
\item
  ``Currently accepting patients'' marker \#67B445
\item
  Desktop navigation hover (darker blue) \#093561
\item
  Desktop primary navigation \#0E4D8F
\item
  Footer background \#0E4D8F
\item
  Footer link text \#B9B5B5
\item
  Form error handling \#C34D36
\item
  Glossary Definition background \#FDFDE2
\item
  Glossary Definition border \#F5F26B
\item
  Link \#0072FF
\item
  Link hover state \#0E4D8F
\item
  Navigation active (current page) link \#A4CAFA
\item
  Navigation background darker \#E5E4E4
\item
  Navigation background lighter \#F3F3F3
\item
  Promo background \#F3F3F3
\item
  Subsite name in header type \#0E4D8F Done 20181208
\end{itemize}

\begin{enumerate}
\def\labelenumi{\arabic{enumi}.}
\setcounter{enumi}{2}
\tightlist
\item
  Check the outliers found in the first box and whisker plot when the
  file ``Example\_Pedigree.csv'' is analyzed. The outliers are exactly
  as described: Data beyond the end of the whiskers are called
  ``outlying'' points and are plotted individually. The jittered points
  are overlayed on top of the boxplot and its outliers. Done 20181112
\item
  Add the ability to select a sex ratio for group formation. Ratios are
  to be female to male starting with 1:1 and progressing by 0.5 on the
  female side, holding the male side at 1. Go up to a ratio of 10:1. The
  progression will look something like 1:1, 1.5:1, 2:1, 2.5:1 \ldots{}
  10:1.
\item
  Add the ability to use harems with 1 male and any number of females.
  One way to do this would be to put any number of males in with the
  females in the list of candidates and then specify that single male
  harems are to be formed with a specified number of groups formed. This
  will allow the select of the best males to be selected from the excess
  male candidates.
\item
  Beth is to contact Mark at least a week prior to the next meeting to
  set up a time to diagnose why her installation of the software is not
  working. Rcpp package was not being replaced during installation and
  update of other packages. This prevented nprcmanager from being
  installed. Beth manually removed the old version of the Rcpp package,
  installed a newer version and then installed nprcmanager and
  rmsutilityr. Mark will do some research to see if others have reported
  similar problems. Working with Beth prompted the development of a
  function that reports the version of the application on the input tab
  so users can easily find out which version they are using. Done
  20181109.
\item
  Until we work on LabKey connectivity again, we have decided to
  postpone Item 2 from two meetings ago: Give user the option to save a
  skeleton configuration file to their home directory if they do not
  have a configuration file.
\item
  We reaffirmed the plan for Mark to finish the tutorials (one for an
  interactive use of the software and one for using the Shiny
  application) and to get a draft of a technical paper written by the
  end of the year.
\item
  Not discussed during the meeting was the need to develop additional
  unit tests to cover all of the new functions created to handle the
  PEDSYS and military formated dates (YYYYMMDD), which look like an
  integer. Most of those unit tests have been made, but additional ones
  are needed to provide full or near full coverage. Done 20181112
\item
  Miscellaneous
\end{enumerate}

\begin{itemize}
\tightlist
\item
  Found that breeding groups being formed included unknown animals that
  had been added as placeholders for unknown parents. Those were removed
  from consideration. Done 20181119
\item
  Worked with Terry Hawkins on 20181119 and 20181120 to make sure the
  LabKey code was working. He had an error in his base URL specification
  so that initial efforts failed and the error message was misleading. I
  have trapped the erorr with a tryCatch function and am sending a
  message to the log file. This needs to be tested.
\end{itemize}

\begin{enumerate}
\def\labelenumi{\arabic{enumi}.}
\setcounter{enumi}{10}
\tightlist
\item
  The next meeting is scheduled for 20181126. Due to conflicts in Mark's
  schedule, we rescheduled for 20181210.
\end{enumerate}

\hypertarget{section-19}{%
\subsection{20181022}\label{section-19}}

\begin{enumerate}
\def\labelenumi{\arabic{enumi}.}
\tightlist
\item
  Allow the use of Ego for ID. Done 20181022
\item
  Remove the display of the change columns. Done 20181103
\item
  Remove old error tabs each time a file browsing occurs. Done 20181103
\item
  List columns that should be there id, sire, dam, sex, birth when
  columns are listed as missing. Done 20181103
\item
  Improve the reporting of multiple errors of the same type. Perhaps
  ``The first 5 records have the following errors: . In total there were
  records with the same type of error. Please check and correct the
  pedigree file.'' Changes are improved over the suggestions above, but
  similar. Done 20181103
\item
  Correct ``Genetic Uniqueness Score'' to ``Genome Uniqueness Score''.
  Done 20181022
\item
  Item 2 from last meeting list: Give user the option to save a skeleton
  configuration file to their home directory if they do not have a
  configuration file.
\item
  During the next meeting, we will work with Beth on clearly defining
  how breeding group formation tab should work. In the mean time, Mark
  will examine the current behavior and code so that he understands
  exactly what is being done.
\item
  The near term goal is to produce a technical paper within the next six
  months that describes the software. Mark will get this started by
  completing two tutorial documents. The first will be a tutorial on how
  to use the major functions from the R console in an interactive mode.
  The second will be a tutorial on how to use the Shiny application. The
  first tutorial should be completed some time in December.
\item
  Amanda and Mark discussed the goal of enhancing the scope of software
  to include the ability to do phenotype -- genotype association
  studies. Amanda has some ideas regarding direction. Mark needs some
  guidance as to what techniques to investigate.
\end{enumerate}

Below are some relevant articles found from a simple web search with the
Google search engine using the search elements: ``genetic association
studies in pedigrees'' *
\url{https://www.stats.ox.ac.uk/~mcvean/gwa4.pdf} *
\url{https://dx.doi.org/10.1186\%2F1753-6561-8-S1-S26} *
\url{https://www.ncbi.nlm.nih.gov/pmc/articles/PMC1459209/} *
\url{https://www.ncbi.nlm.nih.gov/pubmed/12687644} 11. Not mentioned
during the meeting but taken from other correspondence. Mark will
Contact Daniel Nicolalde
(\href{mailto:fdnicolalde@primate.wisc.edu}{\nolinkurl{fdnicolalde@primate.wisc.edu}};
(608) 890-4592; INFORMATICS AND DATA SERVICES) to see if he will assist
in getting the LabKey interface working with the Wisconsin National
Primate Research Center system. 12. The next meeting is scheduled for
November 5, 2018 at 4 PM Pacific Time.

\hypertarget{section-20}{%
\subsection{20180917}\label{section-20}}

\begin{enumerate}
\def\labelenumi{\arabic{enumi}.}
\tightlist
\item
  Change Pedigree browser button label to say ``Trim pedigree based on
  focal animals''. Done 20180917
\item
  Give user the option to save a skeleton configuration file to their
  home directory if they do not have a configuration file.
\item
  Incorporate the error detection functions into the shiny application
  so that if a defective pedigree file is selected by the user a new tab
  is created with a description of the errors detected.
\end{enumerate}

\begin{enumerate}
\def\labelenumi{\alph{enumi}.}
\tightlist
\item
  The user should be moved to the error report tab. Done 20181020
\item
  The error report tab should be placed immediately after the input tab.
  Changed to immediately before the input tab. Done 20181020
\end{enumerate}

\begin{enumerate}
\def\labelenumi{\arabic{enumi}.}
\setcounter{enumi}{3}
\tightlist
\item
  Beth is to try the current (0.3.31) or later version of nprcmanager
  with the pedigree file that has been failing.
\end{enumerate}

\begin{enumerate}
\def\labelenumi{\alph{enumi}.}
\tightlist
\item
  If that file still causes an error, Beth is to send it to Mark.
  Received file from Beth 20181018.
\item
  Mark will change the code to better handle the issues observed, update
  the package, and have Beth test again. All code changes made for all
  known file errors completed 20181020.
\end{enumerate}

\hypertarget{section-21}{%
\subsection{20180820}\label{section-21}}

Meeting was canceled.

Topics I wanted to bring up.

\begin{enumerate}
\def\labelenumi{\arabic{enumi}.}
\tightlist
\item
  Check on ability of Beth and Amanda to install package.
\item
  Demonstrate use of \textbf{qcStudbook} in interactive mode.
\item
  See if I should follow up with users who have had trouble loading a
  file.
\item
  Demonstration to genetics and genomics group
\item
  Breeding group formation. Workflow and goals
\item
  Genetic value analyses. Workflow and goals
\end{enumerate}

\hypertarget{meeting-notes}{%
\subsection{20180711 Meeting Notes}\label{meeting-notes}}

\hypertarget{summary-statistics-tab}{%
\subsubsection{Summary Statistics tab}\label{summary-statistics-tab}}

\begin{enumerate}
\def\labelenumi{\arabic{enumi}.}
\tightlist
\item
  Use default box and whisker plot with red dots for outliers.
\end{enumerate}

\begin{enumerate}
\def\labelenumi{\alph{enumi}.}
\tightlist
\item
  Changes made. Done 2018-07-14
\item
  Hover text to box and whisker plots added. Done 2018-07-14
\end{enumerate}

\hypertarget{pedigree-browser-tab}{%
\subsubsection{Pedigree Browser tab}\label{pedigree-browser-tab}}

\begin{enumerate}
\def\labelenumi{\arabic{enumi}.}
\tightlist
\item
  Change Update breeding colony button to "" Waiting on Amanda. Changed
  to ``Update Focal Animals'' based on instructions from Amanda. Done
  2018-08-04
\item
  Add function to ``Update Breeding Colony'' button to browse for a
  file; add animal IDs to the window where they could be pasted. Started
  2018-07-29 getting file not updating window. Done 2018-08-15
\item
  Change first bullet point to reflect change in function of the
  ``Update Focal Animal'' button. Done 2018-08-15
\item
  Move first bullet point to far right. Done 2018-08-15
\item
  Remove second bullet point. However, provide this information when the
  file has any type of error. This is complete for one error at a time.
  Currently the \textbf{qcStudbook} function does not continue scanning
  the file when an error is found. Done 2018-07-17
\item
  Change ``Display UIDs for partial parentage'' to ``Display UIDs''.
  Consider coming up with a better name than ``UID''. Add hovertext to
  explain what this button does. Proposed explanatory text: ``Unknown
  IDs are created by the application for all animals with only one
  parent. They begin with a capital \textbf{U}.'' Done 2018-08-15
\item
  Change ``Trim pedigree based on specified population'' to ``Trim
  Pedigree''. Add hovertext to explain what this button does. See what
  ``Trim Pedigree'' button does. Does it remove animals? Its intent is
  to reduce the number of animals being examined. It does reduce the
  number by removing all individuals not related to the focal animals.
  Does it change genetic analysis. It does not change the genetic
  analysis for the animals left. Obviously for animals removed, it does.
  The name of this button will likely follow from what the new name for
  ``Update Breeding Colony'' button. Done 2018-08-16
\item
  Add a better description for the Search field. Done 2018-08-16
\end{enumerate}

\hypertarget{error-handling}{%
\subsubsection{Error Handling}\label{error-handling}}

\begin{enumerate}
\def\labelenumi{\arabic{enumi}.}
\tightlist
\item
  Develop a full set of error messages appropriate for all types of
  input files.
\end{enumerate}

\begin{enumerate}
\def\labelenumi{\alph{enumi}.}
\tightlist
\item
  Added improved error detection and reporting for missing required
  columns - Done 2018-07-17
\item
  Added the ability to call qcStudbook interactively with a flag
  (\textbf{reportErrors}) set to TRUE that causes qcStudbook to report
  back a list with all errors detected within the input file. Not all
  errors can be detected in a single pass because some errors preclude
  checking for others. For example, if the birth date column has an
  invalid format, the parental age cannot be checked. If no errors are
  found with \textbf{reportErrors} set to TRUE and NULL value is
  returned. \textbf{reportErrors} defaults to \textbf{FALSE}.
\item
  This emphasizes the need for a good set of corrupt input files for
  testing and unit test development.
\end{enumerate}

\begin{enumerate}
\def\labelenumi{\arabic{enumi}.}
\setcounter{enumi}{1}
\tightlist
\item
  Amanda and Beth will provide me with some files with errors.
\end{enumerate}

\begin{enumerate}
\def\labelenumi{\alph{enumi}.}
\tightlist
\item
  Amanda provided a file with two error types: bad birth date type -
  integer instead of character representation of a date and animal
  appearing as a sire and a dam. Done 20180720
\item
  Both errors were caught, but the first error was only seen after
  correcting the first. It would be better to report all errors and set
  a flag that told the program to stop at the end of parsing the entire
  pedigree.
\item
  Three files were added to the example files:
  \emph{Example\_Pedigree.csv},
  \emph{Pedigree\_File\_Example\_134M\_dam\_removed.csv}, and
  \emph{Pedigree\_File\_Example\_CSV.csv} Done 20180721
\item
  These files need to be documented.
\end{enumerate}

\begin{enumerate}
\def\labelenumi{\arabic{enumi}.}
\setcounter{enumi}{2}
\tightlist
\item
  Refactored some code within \textbf{qcStudbook} to a separate function
  \textbf{unknown2NA}. Done 2018-07-18
\item
  Added bad date string format detection in \textbf{convertDates}, which
  is now \textbf{convertDate}. Also added unit tests for
  \textbf{convertDate}. Done 2018-07-21
\item
  Added ability to handle dates with NA to \textbf{convertDates} Done
  2018-07-22
\item
  Added unit test coverage of \textbf{updateProgress} calls in
  \emph{test\_reportGV.R} Done 2018-07-22
\item
  Corrected a bug and added corollary unit test for
  \textbf{rankSubjects}. Done 2018-07-29
\item
  Added test for missing \textbf{age} column in \emph{orderReport}
  function. This also allows correct ordering when the age column is
  missing. Done 2018-07-29
\item
  Added unit tests for \emph{findLoops} function. Done 2018-07-30
\item
  Added check for class on objects in \emph{convertDates} function,
  because it was failing when objects were already dates. It now passes
  those objects through untouched. Done 2018-08-04
\item
  Added some debug code for updating focal animals (still named breeding
  colony update in code) Done 2018-08-15
\end{enumerate}

\hypertarget{multiple-tabs}{%
\subsubsection{Multiple tabs}\label{multiple-tabs}}

\begin{enumerate}
\def\labelenumi{\arabic{enumi}.}
\tightlist
\item
  Come up with a different word for ``Breeders''. What do you think
  about \emph{Managed Animals}
\end{enumerate}

\begin{enumerate}
\def\labelenumi{\alph{enumi}.}
\tightlist
\item
  Amanda suggested \emph{Focal Animals} -- this is being adopted.
\end{enumerate}

\hypertarget{miscellaneous-items}{%
\subsubsection{Miscellaneous items}\label{miscellaneous-items}}

\begin{enumerate}
\def\labelenumi{\arabic{enumi}.}
\tightlist
\item
  Added automatic generation of a web site
\end{enumerate}

\begin{enumerate}
\def\labelenumi{\alph{enumi}.}
\tightlist
\item
  2018-07-17 first draft
\item
  Need to first move content that has been prepared to the correct
  location. Done 2018-08-16
\end{enumerate}

\begin{enumerate}
\def\labelenumi{\arabic{enumi}.}
\setcounter{enumi}{1}
\tightlist
\item
  Changed read.csv to read.table in code to emphasize we are able to
  read multiple file types.
\end{enumerate}

\begin{enumerate}
\def\labelenumi{\alph{enumi}.}
\tightlist
\item
  2018-07-17
\end{enumerate}

\begin{enumerate}
\def\labelenumi{\arabic{enumi}.}
\setcounter{enumi}{2}
\tightlist
\item
  Amended documentation for the \textbf{addGenotype} function to
  indicate that it is assuming the \emph{genotype} object was opened by
  \textbf{checkGenotypeFile}. 2018-07-17
\item
  Added use of convenience functions \textbf{get\_and\_or\_list} and
  \textbf{is\_valid\_data\_str} from the
  \emph{github/rmsharp/rmsharp/rmsutilityr} repository. 2018-07-21
\item
  Added documentation for data elements \textbf{finalRpt} and
  \textbf{rpt}, which are both created by the \emph{reportGV} function.
  2018-07-29
\item
  Added brief tutorial on how to use \emph{findLoops} function.
  2018-08-04
\end{enumerate}

\hypertarget{questions-that-came-up-after-the-meeting}{%
\subsubsection{Questions that came up after the
meeting}\label{questions-that-came-up-after-the-meeting}}

\begin{enumerate}
\def\labelenumi{\arabic{enumi}.}
\tightlist
\item
  What is the meaning of the following:
\end{enumerate}

\begin{enumerate}
\def\labelenumi{\alph{enumi}.}
\tightlist
\item
  Text under \textbf{Export} button on \emph{Pedigree Browser} tab. ``A
  population must be defined before proceeding to the Genetic Value
  Analysis''.
\item
  Check box above \textbf{Export} button on \emph{Pedigree Browser} tab.
  ``Trim pedigree based on specified population''. Its intent is to
  reduce the number of animals being examined. It does reduce the number
  by removing all individuals not related to the focal animals. Does it
  change genetic analysis. It does not change the genetic analysis for
  the animals left. Obviously for animals removed, it does. Done
  2018-08-16
\end{enumerate}

\hypertarget{meeting-notes-1}{%
\subsection{20180611 Meeting Notes}\label{meeting-notes-1}}

\hypertarget{input-tab}{%
\subsubsection{Input tab}\label{input-tab}}

\begin{enumerate}
\def\labelenumi{\arabic{enumi}.}
\tightlist
\item
  Move Column content into description column separate with colon - Done
\item
  Change Column Name to Allowable Name - Done
\item
  Names have alphanumeric plus "\_``,''-``, and'' ". - Done
\item
  Remove " Other characters have not been tested." - Done
\item
  Change ``Any animals listed as a Sire or Dam that do not have their
  own row or line entry as an Ego will be added.'' to A new row entry
  will be added for any Sire or Dam that do not already have their own
  row as an Ego. - Done
\item
  Change ``Parents will be checked to ensure their own Ego entry is the
  correct sex.'' to ``Animals will be checked to ensure that their sex
  is consistent throughout the file.''" - Done
\item
  Use Allele\_1 Allele\_2 - Done
\end{enumerate}

\hypertarget{summary-statistics-tab-1}{%
\subsubsection{Summary Statistics tab}\label{summary-statistics-tab-1}}

\begin{enumerate}
\def\labelenumi{\arabic{enumi}.}
\tightlist
\item
  Remove background on histograms - Done
\item
  Add box and whisker plot to right of summary statistics histograms -
  Done
\item
  Use Glossary for definition of terms
\end{enumerate}

\hypertarget{meeting-notes-2}{%
\subsection{20180501 Meeting Notes}\label{meeting-notes-2}}

\begin{enumerate}
\def\labelenumi{\arabic{enumi}.}
\tightlist
\item
  The second bullet point under the Input File Handling section of the
  Input tab says ``Please be aware that animals with no parents will be
  treated as founders in these calculations, i.e., sources of new
  genetic variation in the colony.''
\end{enumerate}

I am assuming that this is fine and stated only to keep users aware that
if they have parental information it has to be entered.

\begin{enumerate}
\def\labelenumi{\arabic{enumi}.}
\setcounter{enumi}{1}
\tightlist
\item
  The third bullet point under the Input File Handling section of the
  Input tab says ``Designation of two alleles is required as currently
  there is no accommodation for partial genetic information for an
  individual.''
\end{enumerate}

Is this will be a problem that needs to be addressed?

\begin{enumerate}
\def\labelenumi{\arabic{enumi}.}
\setcounter{enumi}{2}
\tightlist
\item
  Reminder for Mark only: The third bullet point under the Input File
  Handling section of the Input tab says ``If the Age column is
  provided, the program will use the user-specified age.''
\end{enumerate}

Make sure this is true. I may also want to check for internal
inconsistencies.

\begin{enumerate}
\def\labelenumi{\arabic{enumi}.}
\setcounter{enumi}{3}
\tightlist
\item
  In the Input File Handling section of the Input tab the last bullet
  point immediately above the first table says ``Genotype data may be
  supplied within the pedigree file or in a separate genotype file. Only
  two additional columns (first and second) are required when the
  genotypes are provided within the pedigree file.''
\end{enumerate}

This should be explained more fully and the program behavior may need to
be changed. Currently the program assumes the columns first and second
are there and are correctly formed. Currently there is no check to
ensure that first\_name and second\_name data elements are consistent
with first and second data elements.

\begin{enumerate}
\def\labelenumi{\arabic{enumi}.}
\setcounter{enumi}{4}
\tightlist
\item
  In the Input File Handling section of the Input tab and under the
  first table of column names, there is a section describing what is
  done by the qcStudbook function to correct data and create new
  columns.
\end{enumerate}

Should we provide a summary of changes made to the input data such as
changes in column names, sex identifiers, etc., removal of duplicated
rows and columns that are added?

Need name for genotype (first\_name, second\_name) Need new name for
breeders only file type. Reword ``Animals without birth dates are not
considered.'' When minimum parent age is included, animals without birth
dates are not rejected.

Change update breeding colony (pedigree browser) by adding explanatory
text.

\hypertarget{miscellaneous-accomplishments}{%
\subsection{Miscellaneous
Accomplishments}\label{miscellaneous-accomplishments}}

These should be stored elsewhere

\hypertarget{section-22}{%
\paragraph{20180429}\label{section-22}}

Test coverage is 86.32\%.

\hypertarget{section-23}{%
\paragraph{20171004}\label{section-23}}

Unit test coverage is 84.79\%.

\hypertarget{section-24}{%
\paragraph{20170921}\label{section-24}}

Unit test coverage is about 15\%.

\hypertarget{section-25}{%
\paragraph{20170919}\label{section-25}}

We have successfully installed and used \emph{nprcmanager} on Microsoft
(MS) Windows 7, MS Windows 10, and MacOS 10.12.6 running R 3.4.1.

\hypertarget{section-26}{%
\paragraph{20170917}\label{section-26}}

I discovered why the plot of genetic uniqueness scores was not as
expected. It was from an analysis where 0 was the threshold setting,
which I think we should consider removing as an option. I have set the
default to 3, which is what Amanda Vinson's paper indicated that ONPRC
typically used.

I got the logging system integrated into the package. I am using the
package futile.logger (funny name). Note the check box at the bottom of
the side panel on the two attached images. When the Debug on check box
is checked (it is not checked by default), the application writes to a
file in the users home directory named \emph{nprcmanager.log}. Currently
I am only logging events occurring the the \emph{server.R} file, because
that is where I tend to have most of my problems show up. I have
attached an example file in which turned on logging at the debug level
and then read in a couple of pedigree files.

I added code coverage reports to the automated build system running on
Travis-CI.org. Currently we only have 5.68\% of the code being tested
with the unit tests I created thus far. I do not know the code well
enough to know what percentage we will have when I feel comfortable, but
I know most of the functions have no tests at all.

\hypertarget{section-27}{%
\paragraph{20170916}\label{section-27}}

Can now assign known genotypes to individuals and to incorporate that
information into the gene dropping routine. This has been done at the
function level in the \emph{geneDrop} function. The pedigree submitted
via the UI can optionally contain genetic information.

\hypertarget{section-28}{%
\paragraph{20170915}\label{section-28}}

Added a genetic uniqueness plot to the Summary Statistics tab.

\hypertarget{section-29}{%
\paragraph{20170911}\label{section-29}}

I have connected the Travis CI (continuous integration) tool to my
github.com/rmsharp/nprcmanager so that when I check in code it
automatically tries to build it from scratch using packages from CRAN.
Travis CI is now automatically building nprcmanager without errors or
warnings. This puts us considerably closer to being able to offer the
package via CRAN if we want.

\hypertarget{meeting-notes-3}{%
\subsection{20170323 Meeting notes}\label{meeting-notes-3}}

I met with Jack Kent, Deborah Newman, and Charles Peterson about how to
use genetic data in a gene dropping simulation.

From: R. Mark Sharp
\href{mailto:msharp@TxBiomed.org}{\nolinkurl{msharp@TxBiomed.org}}

Subject: Re: Genetic management of colonies

Date: March 23, 2017 at 5:01:07 PM CDT

To: Jack Kent
\href{mailto:jkent@txbiomed.org}{\nolinkurl{jkent@txbiomed.org}}

Cc: Deborah Newman
\href{mailto:dnewman@txbiomedgenetics.org}{\nolinkurl{dnewman@txbiomedgenetics.org}},
Charles Peterson
\href{mailto:charlesp@txbiomed.org}{\nolinkurl{charlesp@txbiomed.org}}

Jack, Debbie, and Charles,

Thank you for meeting with me. I appreciate your help in clarifying the
various stages of the issues associated with providing genetic
management guidance via simulation with partially known MHC data.

\begin{enumerate}
\def\labelenumi{\arabic{enumi}.}
\item
  I will look more closely into what will be needed to add genetic
  information and gene frequency information into the gene dropping
  routines in the kinship2 package. The initial implementation will use
  only a single locus, which should be a sufficient for MHC data

  A. First step will be to ensure we get expected proportions of genes
  in children of individuals when we provide genetic data to all
  founders using a gene frequency based algorithm.

  B. Second step will be to ensure we get expected proportions of genes
  in children when all parental generations have fully known genotypes.

  C. Third step will be to ensure we get expected proportions of genes
  in children when parental generations have partially known genotypes
  and all other genotypes are uninformative.

  D. Fourth step will be to ensure we get expected proportions of genes
  in children when parental generations have partially known genotypes
  and remaining parental genes are determined by gene frequency.
\item
  We will not worry about getting rid of genes in the pedigrees since
  colony managers need only not breed animals that carry unwanted
  alleles.
\item
  I will produce a small number of pedigree drawings using the data
  Debbie has provided to get feedback on preferences and additional
  requirements.
\item
  I will not deal with paternity and maternity issues within the
  pedigrees at this time.
\end{enumerate}

Thanks again.

\end{document}
